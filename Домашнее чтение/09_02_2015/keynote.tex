\documentclass[10pt,a4paper]{article}
\usepackage[utf8]{inputenc}
\usepackage[russian]{babel}
\usepackage{amsmath}
\usepackage{amsfonts}
\usepackage{amssymb}
\usepackage{graphicx}
\author{Агафонова Оксана}
\title{Аналитическое чтение тезисов с пленарных заседаний ACM CCS'13-14}
\begin{document}
\maketitle
\clearpage
\tableofcontents
\clearpage
\section{The Science, Engineering and Business of Cyber Security}
\hspace{0,6cm} Глобальная информатизация в настоящее время активно управляет существованием и жизнедеятельностью государств мирового сообщества, информационные технологии применяются при решении задач обеспечения национальной, военной, экономической безопасности и др. Вместе с тем, одним из фундаментальных последствий глобальной информатизации государственных и военных структур стало возникновение принципиально новой среды противоборства конкурирующих государств – киберпространства. Киберпространство, как «комплекс среды и, как следствие в результате взаимодействия людей, программного обеспечения и услуг в Интернете с помощью технологии устройств и сетей, подключенных к ней, которых не существует в любой физической форме». В современных условиях вопросы кибербезопасности выходят с уровня защиты информации на отдельном объекте вычислительной техники на уровень создания единой системы кибербезопасности государства, как составной части системы информационной и национальной безопасности, отвечающей за защиту не только информации в узком смысле этого слова, но и всего киберпространства.  Противоборство в киберпространстве становится принципиально новой сферой противоборства между государствами. 

Специалисты в области информационных технологий наиболее развитых государств мира единодушно отмечают тот факт, что «государство, контролирующее киберпространство, будет контролировать войну и мир». Вслед за США, стратегии кибербезопасности приняты во многих странах. Киберпространство стало рассматриваться Вашингтоном таким же потенциальным полем боя, как земля, воздух, море и космос. Поэтому США приравнивают акты кибератак к традиционным военным действиям и предусматривают возможность «отвечать на серьезные нападения пропорциональными и справедливыми мерами», вплоть до применения ядерного  оружия.

\section{Exciting Security Research Opportunity: Next-generation Internet}
\hspace{0,6cm}Развитие Интернета стало успешным, превзошел все ожидания, переплетается почти со всеми  аспектами нашего общества и экономики.  Если будут даже незначительные перебои в Интернет все это скажется на различных сферах жизнедеятельности. При этом интернет не является безопасным, хотя и ведутся исследования в этой сфере. Изучается проблема о том какой интернет будет в будущем и что необходимо сделать, чтобы он был безопасным.

\section{The Cyber Arms Race}
\hspace{0,6cm}Когда Интернет только набирал популярность многие высокопоставленные лица проигнорировали его. Они не рассматривали Интернет, как важный критерий для развития. Но спустя некоторое время ситуация кардинально изменилась и политики поняли, как важен Интернет и что его можно применять очень успешно для разных целей, в том числе для слежки за гражданами.

Очень активно стали говорить о Интернет-слежке, после заявлении Эдварда Сноудена о том, что США следит за гражданами с помощью программы PRISM. До некоторого времени эта программа была засекречена, но после утечки информации в СМИ американские политики уже не смогли молчать. Было объявлено, что данная программа контролирует иностранцев на территории США, но все оказалась куда масштабней. Как утверждается, данная программа использовалась не только на территории США, но и для слежки за Европейскими странами и высокопоставленными лицами.
Америка не единственная страна, которая ведет интернет-наблюдения, но у нее есть преимущества. Почти все веб-службы, веб-браузеры, поисковые системы, интернет-сервисы и т.д. производятся в США. 

По мнению автора статьи в какой-то мере слежка за людьми - это хорошо. Например, для выявления "плохих людей", задумавших теракты или преступления. А что же делать людям, которые не задумали ничего плохого, ведь таких большинство. Получается, что надо стараться защитить себя и свою информацию от подобных наблюдений из вне. Тут стоит заметить, что в для защиты от PRISM в настоящее время очень хорошо используется Tor Browser, который значительно повышает шансы избежать слежки за интернет-трафиком и возможность защитить себя от перехвата личной информации.

\end{document}