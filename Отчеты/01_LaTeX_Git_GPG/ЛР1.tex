\documentclass[10pt,a4paper]{article}
\usepackage[utf8]{inputenc}
\usepackage[russian]{babel}
\usepackage{amsmath}
\usepackage{amsfonts}
\usepackage{amssymb}
\usepackage{graphicx}
\author{Агафонова Оксана}
\title{LaTeX}
\begin{document}
\maketitle
\clearpage
\LaTeX{} --- наиболее популярный набор макрорасширений (или макропакет) системы компьютерной вёрстки \TeX{}, который облегчает набор сложных документов. 

Общий внешний вид документа в LaTeX определяется стилевым файлом. Существует несколько стандартных стилевых файлов для статей, книг, писем и т. д., кроме того, многие издательства и журналы предоставляют свои собственные стилевые файлы, что позволяет быстро оформить публикацию, соответствующую стандартам издания.

Во многих развитых компьютерных аналитических системах, например, Maple, Mathematica, Maxima, Reduce возможен экспорт документов в формат *.tex. Для представления формул в Википедии также используется TeX-нотация.

Термин LaTeX относится только к языку разметки, он не является текстовым редактором. Для того, чтобы создать документ с его помощью, надо набрать .tex-файл с помощью какого-нибудь текстового редактора. В принципе, подойдёт любой редактор, но большая часть людей предпочитает использовать специализированные, которые так или иначе облегчают работу по набору текста LaTeX-разметки.

Будучи распространяемым под лицензией LaTeX Project Public License, LaTeX относится к свободному программному обеспечению.
\clearpage
\tableofcontents
\clearpage

\section{Система набора}
\hspace{0,6cm}Главная идея \LaTeX{} состоит в том, что авторы должны думать о содержании, о том, что они пишут, не беспокоясь о конечном визуальном облике (печатный вариант, текст на экране монитора или что-то другое). Готовя свой документ, автор указывает логическую структуру текста (разбивая его на главы, разделы, таблицы, изображения), а LaTeX решает вопросы его отображения. Так содержание отделяется от оформления. Оформление при этом или определяется заранее (стандартное), или разрабатывается для конкретного документа.

Это похоже на стили оформления, которые используются в текстовых процессорах, или на использование стилевых таблиц в HTML.

\section{Возможности}
\hspace{0,6cm}Возможности системы, в принципе, не ограничены (из-за механизма программирования новых макросов). Вот список некоторых возможностей, предлагаемых стандартными макросами и теми, которые можно скачать  с сервера CTAN:
\begin{itemize}
\item алгоритмы расстановки переносов, определения междусловных пробелов, балансировки текста в абзацах;
\item автоматическая генерация содержания, списка иллюстраций, таблиц и т. д.;
механизм работы с перекрёстными ссылками на формулы, таблицы, иллюстрации, их номер или страницу;
\item механизм цитирования библиографических источников, работы с библиографическими картотеками;
\item размещение иллюстраций (иллюстрации, таблицы и подписи к ним автоматически размещаются на странице и нумеруются);
\item оформление математических формул, возможность набирать многострочные формулы, большой выбор математических символов;
\item оформление химических формул и структурных схем молекул органической и неорганической химии;
\item оформление графов, схем, диаграмм, синтаксических графов;
\item оформление алгоритмов, исходных текстов программ (которые могут включаться в текст непосредственно из своих файлов) с синтаксической подсветкой;
\item разбивка документа на отдельные части (тематические карты).
\end{itemize}

\section{Формула нормального распределения}
\hspace{0,6cm}Нормальное распределение, также называемое распределением Гаусса — распределение вероятностей, которое в одномерном случае задается функцией плотности вероятности, совпадающей с функцией Гаусса:
\begin{equation}
\frac{1}{\sigma\sqrt{2\pi}}
   \exp\left(-\frac{(x-\mu)^2}{2\sigma^2}\right)
\end{equation}\
\end{document}